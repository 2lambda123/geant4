\section{Introduction}

% [general inc stuff]
The intranuclear cascade model (INC) was was first proposed by Serber in 1947 \cite{serber47}.  
His noticed that, in particle nuclear collisions the deBroglie wavelenght of the incident particle is 
comparaple (or shorter) than the average intra-nucleon distance.
Hence, the justification for describing the interactions in terms of particle-particle  collisions.


The INC has been succesfully used in the Monte Carlo simulations at intermediate energy region 
since Goldberger made first calculations by hand in 1947 \cite{goldberger48}. 
First computer simulations were done by Metropolis et al. in 1958 \cite{metropolis58}. 
Standard methods in INC implementations were formed when Bertini published his results in 1968 \cite{bertini68}.


%There are several General Monte Carlo simulation toolkits implementing cascade models with various approachies
%such as HETC \cite}{}, and FLUKA \cite{}. 
Having application such as to simulation of hadron calorimetry the INC description of interactions of protons, neutrons and pions with matter is cruisal.

Our presentations describes implementation of INC model in {\sc Geant4} hadronic physics framework \cite{geant4collaboration03}.


%[Geant4 stuff]
Geant4 is a Monte Carlo particle detector simulation toolkit, having applications also in  medical and space
science. 
Geant4 provides a flexible framework for the modular implementation of
various kinds of hadronic interactions. 
Geant4 exploits advanced Software Engineering techniques and Object
Oriented technology to achieve the transparency of the physics
implementation and to this way provide the possibility of validating the
physics results. 

The hadronic models framework is based on concepts of physics
processes and models. While the process is a general concept, models
are allowed to have restrictions in process type, material, element
and energy range.  Several models can be utilized by one model class; for instance, a
process class for inelastic collisions can use distinct models for different energies.
Process classes utilize model classes to determine the
secondaries produced in the interaction and to calculate the momenta
of the particles.  Several model classes for different particles and
energy regimes can be used by the process classes.





