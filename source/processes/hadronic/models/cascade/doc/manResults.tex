\section{Results}
\label{results}

We have redesigned and rewritten HETC and INUCL cascade models using {\sc Geant4} hadronics framework. The cascade code is beeing tested against other simulation codes and experimental data.
The data used in these comparisons are listed in Appendix \ref{manExperiment}.


We have also rewritten INUCL into stand alone software INUCL++ and tested the performance. The code is available from INUCL++ homepage \cite{heikkinen02k}.

We give here summary of results. 
%Each configuration consists of N=1000 events simulated with same energy etc. parameters. 
  
Fig. \ref{pic:z001n002p001e000585c010h000} gives results for (p, D) collisions at $585~MeV$.

Fig. \ref{pic:z001n001p001e000585c010h000} gives results for (p, H) collisions at $585~MeV$.

\subsection{Performance}

Performance tested on un-homogenous Linux (i686) cluster {\em lxplus.cern.ch}.
 
Time ot takes to do cvs checkout for {\sc Geant4} head 8 minutes.


Typical compilation times for the {\sc Geant4} are 50 minutes, 
for the cascade module  ::: minutes, 
for the cascade model only 2 minutes, 
and for the cascade test 15 seconds.

Timing for simulating hadronic cascades :::

\begin{table}
\begin{center}

\begin{tabular}{|l||l|l||}
\hline
&\multicolumn{2}{l|}{p bullet energy}\\
\cline{2-3}
& $0.5~GeV$ & $3~GeV$ \\
\hline\hline
Au    & ::: & ::: \\
Fe    & ::: & ::: \\
Pb    & ::: & ::: \\
\hline
\end{tabular}
\end{center}
\caption{Average time needed for one hadronic cascade.}
\label{tab:timing}
\end{table}


\subsection{Particle multiplicities} 
\label{results:multiplicities}

Multiplities for collisions between $p$  projectile ( $0.5~GeV< E <5.0~GeV$) and aluminium $Al$, iron $Fe$ and lead $Pb$ tragets (at rest) are listed in Figs. \ref{pAlMultiplicity} - \ref{pPbPionMultiplicity}.

\begin{figure}
  \begin{center}
    \leavevmode
    \mbox{\epsfxsize=8cm \epsffile{pictures/pAlMultiplicity.eps} }
       \caption{Particle multiplicities for $(p, ^{27}Al$).}
  \label{pAlMultiplicity}
  \end{center}
\end{figure}

\begin{figure}
  \begin{center}
    \leavevmode
    \mbox{\epsfxsize=8cm \epsffile{pictures/pFeMultiplicity.eps} }
       \caption{Particle multiplicities for $(p, ^{56}Fe$).}
  \label{pFeMultiplicity}
  \end{center}
\end{figure}


\begin{figure}
  \begin{center}
    \leavevmode
    \mbox{\epsfxsize=8cm \epsffile{pictures/pPbMultiplicity.eps} }
       \caption{Particle multiplicities for $(p, ^{82}Pb$).}
  \label{pPbMultiplicity}
  \end{center}
\end{figure}


\begin{figure}
  \begin{center}
    \leavevmode
    \mbox{\epsfxsize=8cm \epsffile{pictures/pPbPionMultiplicity.eps} }
       \caption{Pion multiplicities.}
  \label{pPbPionMultiplicity}
  \end{center}
\end{figure}

\subsection{Particle energies} 
\label{results:energies}

Figs. \ref{pPbProtonEnergy} - \ref{p500MeVPbProtonEnergy} give an summary of cascade products energy-levels.

\begin{figure}
  \begin{center}
    \leavevmode
    \mbox{\epsfxsize=8cm \epsffile{pictures/pPbProtonEnergy.eps} }
       \caption{Average proton energies.}
  \label{pPbProtonEnergy}
  \end{center}
\end{figure}


\begin{figure}
  \begin{center}
    \leavevmode
    \mbox{\epsfxsize=8cm \epsffile{pictures/pPbNeutronEnergy.eps} }
       \caption{Average neutron energies.}
  \label{pPbNeutronEnergy}
  \end{center}
\end{figure}


\begin{figure}
  \begin{center}
    \leavevmode
    \mbox{\epsfxsize=8cm \epsffile{pictures/pPbNucleonExitation.eps} }
       \caption{Average nucleon exitation energy.}
  \label{pPbNucleonExitation}
  \end{center}
\end{figure}


\begin{figure}
  \begin{center}
    \leavevmode
    \mbox{\epsfxsize=8cm \epsffile{pictures/p50MeVPbProtonEnergy.eps} }
       \caption{Proton energy spectrum, when projectile particle (p) had an energy of $0.05~GeV$.}
  \label{p50MeVPbProtonEnergy}
  \end{center}
\end{figure}


\begin{figure}
  \begin{center}
    \leavevmode
    \mbox{\epsfxsize=8cm \epsffile{pictures/p500MeVPbProtonEnergy.eps} }
       \caption{Proton energy spectrum, when projectile particle (p) had an energy of $0.5~GeV$.}
  \label{p500MeVPbProtonEnergy}
  \end{center}
\end{figure}


\subsection{Results from {\em cascade.cc} test}
Results fro, {\sc Geant4} cascade test {\em cascade.cc} and Root batch run {\em cascade.C} are listed in Fig. \ref{cascadeBatch}.

\begin{figure}
  \begin{center}
    \leavevmode
    \mbox{\epsfxsize=13cm \epsffile{pictures/z001n002p001e000585c010h000.eps} }
       \caption{ Results given by {\sc Geant4} cascade test {\em cascade.cc} and Root batch run {\em cascade.C}.}
  \label{cascadeBatch}
  \end{center}
\end{figure}



\begin{figure}
  \begin{center}
    \leavevmode
    \mbox{\epsfxsize=13cm \epsffile{pictures/z001n002p001e000585c010h000.eps} }
       \caption{ Geant4 simulation of proton with $585~MeV$ energy hitting deuterium atom $A(Z = 1,N = 2)$ 10000 times.}
  \label{pic:z001n002p001e000585c010h000}
  \end{center}
\end{figure}



\begin{figure}
  \begin{center}
    \leavevmode
    \mbox{\epsfxsize=13cm \epsffile{pictures/z001n001p001e000585c010h000.eps} }
       \caption{ Geant4 simulation of proton with $585~MeV$ energy hitting hydrogen atom $A(Z = 1,N = 1)$ 10000 times.}
  \label{pic:z001n001p001e000585c010h000}
  \end{center}
\end{figure}

To validate the functionality of the HETC code, several simulations
were carried out. As the other parts of the implementation of HETC
were not yet ready, the evaporation and de-excitation model was run
separately in order to check its functionality. Some elementary tests
were carried out to validate the implementation. 
%Also, the results
%were compared with experimental results when available.

There were no remarkable differences between the results of the first
implementation and the results of the HETC98 code. The differences, which
were of the order of a couple of percent, were thought to arise from
statistical fluctuations. In the comparisons, excitation functions for
all emitted particles as well as for neutrons and protons separately
were considered. Three different elements ($^{16}{\rm O}$, $^{63}{\rm
  Cu}$ and $^{208}{\rm Pb}$) were considered. The graphs of these
tests are, however, omitted here.

%When the results of the second implementation were compared to the
%results of HETC98, larger differences were found. On the excitation
%functions of all emitted particles, seen in
%Fig.~\ref{kaikki}, there are differences of the order of 20~\%
%with $^{16}{\rm O}$. When the excitation functions of neutrons and
%protons are considered (Figs.~\ref{fig:neu} and \ref{fig:pro}), there
%are again such differences with $^{16}{\rm O}$. In other excitation
%functions, except perhaps that of protons with $^{63}{\rm Cu}$, the
%differences are less than 10~\%. There are also large differences in
%the proton excitation function of $^{208}{\rm Pb}$, but this can be
%explained by the fact that the proton emission is a very rare event
%and the related statistical fluctuations are again very large. As to
%the comparison of the kinetic energies of the emitted neutrons and
%protons (Figs.~\ref{fig:kinEneuts} and \ref{fig:kinEprots}),
%differences of less than 20\% were found whenever statistical
%fluctuations were small.

%\begin{figure}[p]
%\centering
%\begin{tabular}{c}
%    \epsfig{file=kuvat/kuva_kaikki.ps,bbllx=50pt,bblly=411pt,bburx=554pt,bbury=605pt,width=0.7\textwidth,clip=}\\
%    \epsfig{file=kuvat/kuva_kaikkierror.ps,bbllx=50pt,bblly=323pt,bburx=555pt,bbury=463pt,width=0.7\textwidth,clip=} \\
%\end{tabular}\\
%    \caption{ Average multiplicity of all evaporated particles
%      of $^{16}{\rm O}$, $^{63}{\rm Cu}$ and $^{208}{\rm Pb}$
%      simulated and compared with the HETC98 code. The obtained
%      average multiplicities and the relative difference (\%) are
%      shown.}
%      \label{fig:kaikki}
%\end{figure}

%\begin{figure}[p]
%\centering
%\begin{tabular}{c}
%    \epsfig{file=kuvat/kuva_neu.ps,bbllx=50pt,bblly=411pt,bburx=554pt,bbury=605pt,width=0.7\textwidth,clip=}\\
%    \epsfig{file=kuvat/kuva_neuerror.ps,bbllx=50pt,bblly=323pt,bburx=555pt,bbury=463pt,width=0.7\textwidth,clip=} \\
%\end{tabular}\\
%    \caption{ Average multiplicity of evaporated neutrons
%      of $^{16}{\rm O}$, $^{63}{\rm Cu}$ and $^{208}{\rm Pb}$
%     simulated and compared with the HETC98 code. }
%      \label{fig:neu}
%\end{figure}


%\begin{figure}[p]
%\centering
%\begin{tabular}{c}
%    \epsfig{file=kuvat/kuva_pro.ps,bbllx=50pt,bblly=411pt,bburx=554pt,bbury=605pt,width=0.7\textwidth,clip=}\\
%    \epsfig{file=kuvat/kuva_proerror.ps,bbllx=50pt,bblly=323pt,bburx=555pt,bbury=463pt,width=0.7\textwidth,clip=} \\
%\end{tabular}\\
%    \caption{ Average multiplicity of evaporated protons
%      of $^{16}{\rm O}$, $^{63}{\rm Cu}$ and $^{208}{\rm Pb}$
%      simulated and compared with the HETC98 code. }
%     \label{fig:pro}
%\end{figure}

%\begin{figure}[p]
%\centering
%\begin{tabular}{c}
%    \epsfig{file=kuvat/kuva_kinEneut.ps,bbllx=50pt,bblly=411pt,bburx=554pt,bbury=605pt,width=0.7\textwidth,clip=}\\
%    \epsfig{file=kuvat/kuva_kinEneuterror.ps,bbllx=50pt,bblly=511pt,bburx=555pt,bbury=592pt,width=0.7\textwidth,clip=} \\
%\end{tabular}\\
%    \caption{ Kinetic energies of evaporated neutrons of
%      $^{16}{\rm O}$, $^{63}{\rm Cu}$ and $^{208}{\rm Pb}$
%     simulated and compared with the HETC98 code. }
%      \label{fig:kinEneuts}
%\end{figure}

%\begin{figure}[p]
%\centering
%\begin{tabular}{c}
%    \epsfig{file=kuvat/kuva_kinEprot.ps,bbllx=50pt,bblly=411pt,bburx=554pt,bbury=605pt,width=0.7\textwidth,clip=}\\
%    \epsfig{file=kuvat/kuva_kinEproterror.ps,bbllx=50pt,bblly=511pt,bburx=555pt,bbury=592pt,width=0.7\textwidth,clip=} \\
%\end{tabular}\\
%    \caption{ Kinetic energies of evaporated protons of
%      $^{16}{\rm O}$, $^{63}{\rm Cu}$ and $^{208}{\rm Pb}$
%      simulated and compared with the HETC98 code. }
%      \label{fig:kinEprots}
%\end{figure}



%\section{HETC++ documentation}%:::






