\section{Introduction to {\sc Geant4} }

% {\it The exploitation of hadronic final states has played a key role in the
% success of all recent HEP collider experiments. It will also be one of
% the key issues during the LHC era. }

Geant4 is a Monte Carlo particle detector simulation toolkit for
various computing platforms and en\-vi\-ron\-ments\cite{G4UserRequirements}.
In order to match the needs of the LHC in high energy physics
community. Other requirements are also set by medical and space
science applications. Geant4 has been developed by a world-wide collaboration of over 40
institutions~\cite{wellisch99,MoURD44}.  
The objectives of the Geant4 development project are to redesign and
re-engineer the major CERN software tool Geant3.21 for an Object Oriented (OO) environment. 

Geant4 collaboration websites \cite{webG4INFN}.
% {\it
% \begin{itemize}
% %\item general overview of Geant4 and its OO development (G3.21)
% \item uses of hadronic models and their status (quite inaccurate)
% \item requirements from the LHC
% \item abstract interfaces??
% \item check: HPW article, G4 web page
% \end{itemize}
% }

\subsubsection{Hadronic Models Framework in Geant4}

%\subsection{Hadronic physics in {\sc Geant4} }


%Parametrization
%(GEISHA), semi-empirical, teoretical 

Geant4 exploits advanced Software Engineering techniques and Object
Oriented technology to achieve the transparency of the physics
implementation and to this way provide the possibility of validating the
physics results. The stringent requirements~\cite{wellisch99} of the
LHC and other future experiments require from the hadronic simulation framework a high
level of flexibility, customizability, extendibility and transparency,
which can be attained only by a systematic OO software engineering
approach. Easy maintenance and distributed development are also
important issues for the framework design.

The hadronic models framework is based on concepts of physics
processes and models. While the process is a general concept, models
are allowed to have restrictions in process type, material, element
and energy range.  Several models can be utilized by one model class; for instance, a
process class for inelastic collisions can use distinct models for different energies.
% Geant4 provides a flexible framework for the modular implementation of
% various kinds of hadronic interactions.  There are distinct process
% classes for different type of interactions (i.e. elastic and
% inelastic).  Process classes utilize model classes to determine the
% secondaries produced in the interaction and to calculate the momenta
% of the particles.  Several model classes for different particles and
% energy regimes can be used by the process classes. 
A simplified UML
diagram of the hadronic models framework is presented in

%Fig.~\ref{fig:HadronicModelsFramework}.
%\begin{figure} [h!tb] 
%\centering
%    \leavevmode
%     \epsfig{file=kuvat/kuva_HadronicModelsFramework.ps,bbllx=55pt,bblly=400pt,bburx=565pt,bbury=755pt,width=0.8\textwidth}
%     \epsfig{file=kuvat/arti_GENeps.ps,bbllx=55pt,bblly=400pt,bburx=565pt,bbury=755pt,width=\textwidth}
%        \caption{Simplified UML diagram describing the hadronic
%  processes and models framework.  Methods and variables have been
%  left out as well as the cross-section classes.
%Specific processes and models can be derived from G4Hadron...Process 
%and G4...Interaction classes, respectively. \label{fig:HadronicModelsFramework}}
%\end{figure}

Geant4 process classes contain two kinds of important methods used in
tracking: {\it GetPhysicalInteractionLength} (GPIL) and {\it
  DoIt}.  The {\it GPIL} method gives the step length from the current
space-time position to the next space-time point where the {\it DoIt}
method is invoked to describe what happens in the interaction.  It
describes the change of energy and momentum direction, change of
position and secondary tracks by returning an instance of the class {\it G4VParticleChange}.

All physics process classes are derived from the class {\it G4VProcess}. It
has three pure virtual {\it DoIt} methods: {\it PostStepDoIt},
{\it AlongStepDoIt} and {\it AtRestDoIt}.

%\scriptsize
%{\it
%\begin{itemize}
%\item G4VParticleChange* PostStepDoIt(const G4Track\& track, const
%  G4Step\& stepData)
%\item G4VParticleChange* AlongStepDoIt(const G4Track\& track, const
%  G4Step\& stepData){, \rm and }
%\item   G4VParticleChange* AtRestDoIt(const G4Track\& track, const
%  G4Step\& stepData).
%\end{itemize}
%}
%\normalsize
%\noindent

%and also three similarly named and 
These methods return {\it G4VParticleChange} classes. 
{\it G4VProcess} also has corresponding pure virtual GPIL
methods, which return a {\it G4double} for the interaction length. There are
also some other pure virtual methods common for all physics processes.
%like the {\it IsApplicable} method, which tells whether the process
%object is applicable to the particle type or not.

For simple processes, there are base classes such as {\it
  G4VRestProcess}, {\it G4VContinuous\-Process} and {\it G4VDiscrete\-Process}
available. In these classes only one of the above mentioned {\it DoIt}
methods is active. The {\it DoIt} method returns an instance of
``Particle Change'' class (derived from {\it G4VParticleChange}),
which is responsible for updating the step class {\it G4Step} used in
tracking.

All hadronic process objects are derived from the abstract class {\it
  G4Hadronic\.Process} (which itself is derived from {\it
  G4VDis\-crete\-Process} and {\it G4VProcess}), and have one or more
cross-section data sets associated with them. These objects
encapsulate methods and data for calculating total cross-sections for
a given process. The default cross-sections can be overridden in whole
or in part (for certain materials and energy regimes) by the user.
Thus the cross-sections and physical models are implemented independently.

The Geant4 hadronic model framework allows flexible use of several models
without the need for the implementation of a special interface. 
This way, highly specialised models can be easily used in the same
application together with more general code. Forexample, models wich
are valid only for one material and particle and applicable only in a restricted energy range.

%At the end of 1999, there were no properly tested hadronic cascade simulation models in Geant4. 
%Several models were implemented, but not  yet thoroughly tested. 
Geant4 models can be divided into parametrisation driven, data driven and
theory driven models~\cite{privAmelin, wellisch99}.
In parametrisation driven models, the existing data for hadronic
reactions and energies is parametrized for generating the final
state. In Geant4, inelastic scattering models based on this paradigm are
available for low and high particle energies.

Data driven models are typically used in Geant4 for the simulation of
nuclear low energy neutron scattering.  It is also used in the
simulation of the absorption of some particle coming to rest. When the data
coverage is not sufficient other types of models are used.

Theory driven models are available in Geant4 for inelastic scattering
in a first implementation, covering the full energy range of LHC
experiments (up to $\rm14~TeV$). Theory driven models are used to
extract the missing cross-sections from the measured ones, or at high
energies to predict the cross-sections by using the Regge theory.


% The hadronic cascade simulation models can be
% divided~\cite{wellisch99} into the following three categories:
% \begin{itemize}
% \item parametrization driven models \ldots
% \item data driven model \ldots
% \item theory driven models \ldots
% \end{itemize}
