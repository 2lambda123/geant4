
\section{INUCL model overview}

\subsection{Background}
Originally the INUCL code was published in 1983 \cite{stepanov}.
This code, written with Fortran has been used in [:::] \cite{:::}.

Comprison of results between INCUL and LAHET, CEM95, HETC, CASCADE,
YIELDX, and ALICE code are presented in Refs. \cite{titarenko99a}.

Year 2000 a aroject to implement INUCL into Geant4 hadronic physic
module using C++ wal launched. 

New now  works also as standalone c++ software called INUCL++.

\subsection{Summary of INUCL model features}
\begin{itemize}
\item Originally INUCL was designed as a particle - nucleus interaction simulation block for the particle - target interaction simulation program PHOENIX. It produces an exclusive approach to simulating events with reasonable performance.
\item INUCL is based on N. Stepanov Ph.D. thesis, ITEP, Moscow, 1990.
Also, contribution Vladimir D. Kazaritsky.
\item Now we have standalone F77 based INUCL code and INUCL++ written in C++.
INUCL++ is now written using Geant4 coding style and integration to hadronic models in in progress. Problem: speed is five times slower.
\end{itemize}

\subsubsection{INUCL models}
\begin{itemize}
\item Intranuclear cascade
\item Precompound decay (exiton master equation) 
\item Evaporation (Weisskopf - Ewing) 
\item Fission (phenomenological model, incorporating some features of the fission statistical model)
\end{itemize}

\subsubsection{Particles treated}
\begin{itemize}
\item Range of targets allowed arbitrary.
\item Range of projectiles allowed p, n, pi and nuclei.
\item From a few Mev to 10 GeV for n, p, pi and up to about 100 MeV / nucleon for nuclei.
\end{itemize}

\subsubsection{Cross sections}
\begin{itemize}
\item Total inelastic cross section has to be taken from outside to normalize all data. 
\item Total reaction cross-sections [mbarn] were calculated by J.R. Letaw's formulae 

%$45 A^0.7 (1+0.016 sin(5.3-2.63 log10(A)))^(1-0.62 exp(-E / 200) sin(10.9 E^(-0.28)))$
 
\item Ref.: S. Pearlstein, The Astrophysical Journal, 346: 1049-1060, 1989 November 15
\item Fermi energy calculated in a local density approximation.
\end{itemize}

Nuclear density distribution are derived from the Re(Vopt( r)) distribution. In cascade part, nucleus is divided into a finite number of zones with constant density.


%nuclear radius parameterization: By the definition R(A) is derived from eq. Den(R(A)) = 0.01*Den max
\subsubsection{Nucleon nucleon cross-sections}
\begin{itemize}
\item Parametrizations based on the experimental data (ED) are used. 
\item They are energy and isospin dependent. 
\item The parameterizations described in ([1] Barashenkov V.S., Toneev V.D. High Energy interactions of particles and nuclei with nuclei. Moscow, 1972 
%(in Russian, but there is an English translation)) are used.

\item Pauli exclusion in the INC: Simulated particle-particle interaction is accepted only for secondary nucleons which have $E_n > E_f$.

\item Nuclear density effects are recalculated after each step

\item Cascade is stopped when all the particles, which can escape the nucleus, do it. Then conformity with the energy - conservation law is checked and the given event is accepted, if $E_{exitation} > E_{cut} \approx $a few $MeV$.

\item For nucleons binding energies are calculated using mass formula. For pions Vopt is taken to be constant (about 7 MeV).
\end{itemize}

%What criteria for p-h excitation? is the next phase precompound or compound'? Only pions. The next phase is precompound. Initial conditions are defined during the cascad phase: p -number of "particles", i.e. nucleons, which can not escape the nucleus and have too small interaction probability; h - number of "holes" = number of nuclear nucleons involved in the cascade; energy - momentum of the exiton system derived from the conservation law.

\subsubsection{Precompound phase}
%, describe the PE model used, parameters, i.e., partial state densities, transition rates? 
\begin{itemize}

\item Main parameters are taken from (Ribansky I. et al, Nucl.Phys.,1973, A205, p.545 (level densities); Kolbach.C., Z.Phys.,1978, A287, p.319 (matrix elements)). 
%(Only N -> N, N -> N + 2, N -> N -2, N -> N - 1 channels are treated.)
\item The angular distribution is isotropic in the frame of rest of the exiton system.
\end{itemize}

%Describe parameters used: level densities, inverse cross-sections or transmission coefficient, choice of optical model parameters if relevant (or reference to source), range of excitations allowed, inclusive or exclusive results? Weisskopf-Ewing evaporation in competition with fission. Emissions of n,p,d,t,He3,He4,gamma is allowed. Level densities derived from exp.data are used. Angular momentum and spin dependence are not included. Other parameters are the same as in ([1], see 5a.) Fermi breakup is allowed onlyin some extreme cases, i.e. for light nuclei and E(exitation) > 3.*Eb. Only the total nuleus decay into neutrons and protons is treated.


\begin{verbatim}
particles p, n, pi, D, T, He3, He4,\gamma
(INUCLN with neutrino 31.3.98)
pion aborption
interaction crosssections
(all data for (N, N) and (pi, N) interactions (dn/dsigma, d3sigma/d3p, 
            partial multiplicity for npi<=5  error 10-20\%)
pre-equilibrium exiton model 
\end{verbatim}

\subsection{Intranuclear cascade model}

\subsubsection{Nucleon dencity in the atom}

Halo nucleus such as $^{11}Li$ are not modelled.

\subsubsection{Impulse distribution}

\subsubsection{Distribution of potential energy}

\subsubsection{Quantum effects}
Pauli exclusion principle

\subsubsection{Description of inc}


\subsection{Fission}

\subsection{Pre-equilibrium}

\subsection{Evaporation}
Exited nuclei cools further trough the emission of gamma radiation.
If simulation is detailed,
the emitted gamma rays contain information about the cooling route 
and region the nucleus is passing trough.

\subsection{From INUCL to INUCL++}

\subsubsection{Design and analysis}
We desided to make separate INUCL++ implementations one as an stand
alone with spesific cross-section data and particle definition, and
another iplementation that re-uses Geant4 classes.
So, requirements for the INUCL++ came mainly from Geant4.
Coding style and organization should follow those used in Geant4.


Interace-classes to Geant4 hadronic models define the separation of
cascade, fission, pre-equilibrium, and evaporation. Thus INUCL++
implementation in Geant4 consists of four separate modules.

\subsubsection{Implementation}
Implementation of INUCL++ in {\sc Geant4}

\subsubsection{INUCL++ as a standalone program}
