\section{HIP GEANT4 ACTIVITIES}
\label{report:project}.

\subsection{General overview}

Project outcomes are {\bf fully documented} in {\sc Geant4} cascade homepage \cite{heikkinen02m},
in web page for INUCL code \cite{heikkinen02k} and for HETC  \cite{heikkinen02j}. 


Publications during the project are numerious. Papers \cite{heikkinen00a}, talks : \cite{heikkinen02i}, \cite{heikkinen02h}, thesis work or spacial assignments \cite{lampen00a} ,\cite{tillikainen01a}.


General:
\begin{itemize}
\item In 2001 HIP joined as participating institute the world wide Geant4 collaboration
\item Responsibility for development and maintenance of the nuclear evaporation and intra-nuclear cascade processes. 
\item Two major codes HETC and INUCL are to be implemented and developed for Geant4 hadronic processes.
\end{itemize}

Research activities:
\begin{itemize}
\item Participation to Geant4 TSB and CB meetings
\item The conversion of the hadronic evaporation processes of HETC code to Geant4 was completed in 2001
\item Good progress was made in the Object-Oriented implementation of the intra-nuclear cascade processes

\item An object oriented model of INUCL was prepared, containing models for intra-nuclear cascade, pre-equilibrium state, fission and evaporation 
\item Implementation of  INUCL++ as a standalone software was finalized

\item Prototype for cascading framework was prepared using experience gained from INUCL OO design
\end{itemize}


Inside HIP LHC Programme Software and Physics project Geant4 activities are under catecory {\it Simulation and event recosntruction}.
Geant4 represents one of the largest and most ambitious projects of geographically-distributed software development and large-scale object-oriented systems.
In 2001 HIP joined as participating institute the worl wide Geant4 collaboration with the major responsibility for development and maintenance of the nuclear evaporation and intra-nuclear cascade processes. 


We are working with intranuclear cascade, pre-equilibrium, fission, and evaporation models. The original codes HETC and INUCL are providing implementations of these models. We re-write the models using OO methds and {\sc Geant4} hadronic physics interfaces.
Our contribution in TSB and CB will be focused in near future to initiatives and coordination of  hadronic cascade framework inside Geant4 hadronic framework.
The conversion of the hadronic evaporation processes of HETC code to Geant4 was completed in 2001 and good progress was made in the Object-Oriented implementation of the intra-nuclear cascade processes.


For another important nuclear Maonte Carlo code, INUCL, an object oriented model was prepared, containing models for intra-nuclear cascade, pre-equilibrium state, fission and evaporation. 
The stand-alone version INUCL++ is comparable to codes such as HETC and LAHET \cite{titarenko99}. 
Architecture of INUCL++ intra-nuclear model plays an important role in our Geant4 work, since we use it also for HETC cascades modelling and as an platform to form a general framework for cascading model.
Year 2002 started with implementation of  INUCL++ as a standalone software.
A preliminary version of architecture using INUCL++ intra-nuclear model. It also integrates HETC cascades and utilites Geant4 hadronic models frameworks.


\subsection{HIP Geant4 activities 2001-2002 }


The year 2002 of the cascade codes is planned as follows:
\begin{itemize}
\item {\bf June}: Release first version of cascade for G4 testing 
\item {\bf July} major development in inc, and indocumentation 
%\item {\bf August} first results and documentation from pre-equilibrium
%\item {\bf September} first results from inc, first relese of pre-equilibrium
\item {\bf October} inc development. Tests should give an overview of of cascade model performance. 
\item {\bf November} testing 
\item {\bf December} testing, documentation and first release.
\end{itemize}

Status of the cascade codes (June 2001):
\begin{itemize}
\item {\bf general status:} we are behind scedule, but we
still hope to provide HETC++ into Geant4 in 2002
\item {\bf evaporation:} ready, documentation will be brovided next
\item {\bf pre-quilibrium:} raw translation done, OOD tuning and testing next
\item {\bf inc:} raw ranslation coming up, next we will build OOA\&D
using this raw translation
\item {\bf documentation:} very preliminary, in july we will have
first HETC++ documentation
\item {\bf testing:} evaporation tested, othervise just infividual
routines tested, next we will set up sandard test suite
\item {\bf results:} results from evaporation tests are compared agains
Fortran version. Module works well. 
\item INUCL cascade model introduced to Geant4. Testing follows.
\item Significant progress on INUCL pre-equilibrium and fission models. We are ready for full interface integration and testing.
\item we will continue testing INUCL++ against previous versions.
\item we reorganize the code is sutch manner that integration to Geant4
  is smooth
\item before year 2002 provide INUCL++ as a stanalone program, 
\item we plan provide it durong 2003 as an hadronic cascade modules  (INC, pre-equilibrium, evaporation, fission) in Geant4. 
\end{itemize}

%Othevise no results, 
%Next results coming in august form pre-equilibrium module. 
%In september first results from inc.
%\item INUCL cascade model introduced to Geant4
%\item Fully integration of HETC intra nuclear cascade -model and Geant4.
%\item Significant progress on HETC and INUCL pre-equilibrium and fission models.
%\item Introduction of general cascade framework to Geant4 hadronic models
%This effort will continue during spring 2001 and our aim is
%to have full HETC code for beta testing before summer 2001.
%Testing and development of implementation is expected to end before
%year 2002.

