\section{PRACTICAL GUIDES}

\subsection{Starting with {\sc Geant4}}
 
Visit \htmladdnormallink{\textsf{http:// wwwinfo.cern.ch/ asd/ geant4/}}{http://wwwinfo.cern.ch/asd/geant4/}
and study downloading documents. The quidance there should be enough, but here follows general outline, though.


You can start {\sc Geant4} -use from the scratch, with following steps:
 
\begin{enumerate}

\item Prepare full Linux (A PC and even a Mac is a possibility.) environment (with minimum $\sim 1~GB$ free disk space)
 
\item Set numeric libraries. {\sc Geant4} needs CLHEP-1.X library.
Download it from \htmladdnormallink{\textsf{http:// wwwinfo.cern.ch/ asd/ lhc++/ clhep/}}{http://wwwinfo.cern.ch/asd/lhc++/clhep/}. , and follow installation instructions. 
Run tests in {\sf CLHEP/test} directory with command {\sf gmake})
 
\item Set visualization packages you want use. Choises are listed in {\sc Geant4} documentation.
I use Mesa (Dowload free OpenGL from \htmladdnormallink{\textsf{http:// www.mesa3d.org/}}{http://www.mesa3d.org/} , follow instructions and compile libraries and demos. 
Run prgrams in samples and demos directories. 
Also I've been using VRMLView Pro software to visualise geometries {\sf bash\$ vrmlview g4.wrl \&}
 
\item Set environment variables as advised in geant4 installation guide
\htmladdnormallink{\textsf{http:// wwwinfo.cern.ch/ asd/ geant4/ G4UsersDocuments/ UsersGuides/ InstallationGuide/ html/ UnixMachines/ unixMachines.html}}{http://wwwinfo.cern.ch/asd/geant4/G4UsersDocuments/UsersGuides/InstallationGuide/html/UnixMachines/unixMachines.html}

At least variables {\sf G4SYSTEM, G4INSTALL, CLHEP\_BASE\_DIR} are reguired.
 
\item Do compilation as advised in installation guide
 
\item Study and compile test programs in {\sf \$G4INSTALL/exaples} directory

Example: 
\begin{verbatim}
cd N01, 
gmake
$G4BINDIR/exampleN02
Idle> help (study command options carefully)
\end{verbatim}

\item Select the most apropriate sample progam, and modify it to your own use
\item You are now ready to start your own studies. Have fun. 

 \end{enumerate}                       


%show g4 general installlation
%explain recursive gmake
%show gamma_ray general structure
% show src files PhysicsList, DetectorConstruction, TrackerROGeometry, PrimaryGeneratorAction, TrackerSD, EventAction 
%run example $G4BINDIR/GammaRayTel
%show intro texts
%show help (/particle/list, /process/list, show payload)
%run ( /tracking/verbose 1, /gun/particle gamma, /gun/energy 200 MeV, /run/beamOn 1)
%show outfile
%show vrlmview g4.wrl
