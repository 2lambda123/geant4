\section{Geant4 cascade model}

In inelastic particle nucleus collision a fast phase ($10^{-23} - 10^{-22} s$) of INC results to highly exited nucleus, 
and is followed possible by fission and pre-equilibrium emission. A slower $10^{-18} - 10^{-16} s$ compound nucleus phase follows with evaportaion.
A Boltzman equation must be solved to treat physical proces of collision in detail.
 
Intranuclear cascade (INC) model developed by Bertini \cite{bertini68, bertini69, bertini71}, solves the Boltzman equation on the average.
The model had been implemented in several codes such as HETC \cite{alsmiller90}. 
Our model is based on re-engineering of INUCL code  \cite{titarenko99a}.
Models included are Bertini intranuclear cascade model with exitons, pre-equilibrium model, simple nucleus explosion model, 
fission model, and evaporation model. 

Nuclear model consist of a three-region approximation to the continuously changing density distribution of nuclear matter within nuclei.
Reletivistic kinematics is applied throughout the cascade.
Cascade is stopped when all the particles, which can escape the nucleus, do it. 
Then conformity with the energy - conservation law is checked.


\subsection{Model limits}

Particles treated are proton, neutron, pions, photon and nuclear isotopes.
Bullet particle can be proton, neutron or  pion.
Range of targets allowed is arbitrary.

The necessary condition of validity of the INC model is $\lambda_{B} / v << \tau_{c} << \Delta t$, 
where $\delta_{B}$ is the de Broglie wavelenth of the nucleons, v is the average relative N-N velocity and $\Delta t$ is the time interval between collisions.
So the physical foundation comes approximate at energies less than $200 MeV$, and needs to be supported with pre-quilibrium model.
also at energies higher than $\approx$ 10 GeV) the INC picture breaks down.
Model has been tested against experimental with bullet kinetic energy  between 100~MeV and 5~GeV.

\subsection{Intranuclear cascade model}

Basic steps of the INC model are summarised below:

\begin{enumerate}
\item The spatial point, where the incident particle enters, is selected uniformly over the projected area of the nucleus.
\item Total particle-particle crossections and region-depenent nucleon densities are used to select a path lenght for the projectile particle.
\item The momentum of the struck nucleon, the type of reaction and four momentum of the reaction products are determined.
\item Exiton model is updated as the cascade proceeds.
\item If Pauli exclusion principle allows and $E_{particle} > E_{cutoff}$ = 2~MeV, step (2) is performed to transport the products.
\end{enumerate}

After INC, the residual excitation energy of the resulting nucleus is used as input for non-equilibrium model.

\subsection{Nuclei model}

Some of the basic features of the nuclei are:

\begin{itemize}
\item The nucleons are assumed to to have a Fermi gas momentum distribution. Fermi energy calculated in a local density approximation i.e. 
Fermi energy is made radius dependent with Fermi momentum $p_{F}(r) = (\frac{3 \pi^2 \rho(r)}{2})^\frac{1}{3}$.
%\item Nuclear density effects are recalculated after each step.
\item Nucleons binding energies (BE) are calculated using mass formula.
A parametrization of the nuclear binding energy uses combination of Kummel mass formula, and
experimental data. Also, asymptotic high temperature mass formula is used if it's impossible to use experimental data.
\end{itemize}

\subsubsection{Initialization}
The initialization phase fixes of nucleus radius  and momentum according to Fermi gas model.

If target is Hydrogen (A = 1) direct particle-particle collision is performed, 
and no nuclear modelling is used. 

If $1 < A < 4$, a nuclei model consisting one layer with radius of 8.0 fm is created.

For $4 < A < 11$, nuclei model is composed of three consentric spheres $i = \{1, 2, 3\}$ with radius
$$r_{i}(\alpha_{i}) = \sqrt{C_{1}^{2} (1 - \frac{1}{A}) + 6.4} \sqrt{-log( \alpha_{i})}$$

Here $\alpha_{i} = \{0.01, 0.3, 0.7\}$ and $C_{1} = 3.3836 A^{1/3}$

If $A > 11$, nuclei model with three consentric spheres is also used. The shere radius is now defined as:
$$r_{i}(\alpha_{i}) =  C_{2} \log({\frac{1 + e^{- \frac{C_{1}}{C_{2}}}}{\alpha_{i}} - 1}) + C_{1}$$

where $C_{2} = 1.7234$.

The potential energy V for nucleon N is
$$ V_{N} = \frac{p_{F}^2}{2 m_{N}} + BE_{N}(A, Z)$$

where $p_f$ is a Fermi momentum and BE a binding energy. 
 
Impulse distribution in each region follows Fermi distribution with zero temperature.

\begin{equation}
 f(p) = c p ^2
 \end{equation}

 where

 \begin{equation}
 \int_0^{p_F} f(p) dp = n_{p}   or  n_{n}
 \end{equation}

 where $n_p$ and $n_n$ are the number of protons or neutrons in region.
 $P_f$ is impulse corresponding the Fermi energy

 \begin{equation}
 E_f = \frac{p_F^2}{2 m_N} = \frac{\hbar^2}{2 m_N}(\frac{3 \pi^{2}}{v})^\frac{2}{3}
 \end{equation}
 
 which depend on the dencity $n/v$ of particles. 
 and which is different for each particle and each region. 

\subsubsection{Pauli exclusion principle}
Pauli exclusion principle forbids interactions where the products would be in occupied states.
Following an assumption of completely degenerate Fermi gas, the levels are filled from the lowest level.
The minimum energy allowed for the product of collision correspond to the lowest unfilled level of system, which is the Fermi energy in the region. 
So in practice, Pauli exclusion prinsiple is taken into account by accepting only secondary nucleons which have $E_N > E_f$.


\subsubsection{Cross sections and kinematics}

Path lengths of nucleons in the nucleus are sampled according to the local density and to free N-N cross sections.
Angles after collisions are sampled from experimental differential cross sections.
%{\sc Geant4} cascade model uses tabulated cross-sections.
Tabulated total reaction cross-sections are calculated by Letaw's formulation \cite{letaw83, letaw93, pearlstein89}.
%:::$45 A^0.7 (1+0.016 sin(5.3-2.63 log10(A)))^(1-0.62 exp(-E / 200) sin(10.9 E^(-0.28)))
For N-N cross-sections parametrizations based on the experimental energy and isospin dependent data. 
The parameterization described in \cite{barashenkov72} is used. 

For pion the INC crossections are privded to treat elestic collisions and following inelstics channels
$\pi^{-}$n $\rightarrow$ $\pi^{0}$n, $\pi^{0}$p $\rightarrow$ $\pi^{+}$n and $\pi^{0}$n $\rightarrow$ $\pi^{-}$p.
Multiple particle production is also implemented.

Pion absorption cannels are 
$\pi^{+}$nn $\rightarrow$ pn, $\pi^{+}$pn $\rightarrow$ pp, 
$\pi^{0}$nn $\rightarrow$ X , $\pi^{0}$pn $\rightarrow$ pn,      $\pi^{0}$pp $\rightarrow$ pp, 
$\pi^{-}$nn $\rightarrow$ X , $\pi^{-}$pn $\rightarrow$ nn , and $\pi^{-}$pp $\rightarrow$ pn.
\subsection{Pre-equilibrium model}

{\sc Geant4} cascade model implements exiton model proposed by Griffin \cite{griffin66, griffin67}.
The his model nucleon states are characterized by the number of exited particles and holes (the exitons).
INC collisions give rise to a sequence of states characterized by increasing exciton number, eventually leading to a equilibrated nucleus.
For practical implementation of exiton model we use parameters  from \cite{ribansky73}, (level densities) and \cite{kalbach78}  
(matrix elements).

In exiton model the posible selection rules for a particle-hole configuarations in the cource of the cascade are:
$\Delta p = 0, \pm 1$  $\Delta h = 0, \pm 1$  $\Delta n = 0, \pm 2$,
where p is the number of particle, h is number of holes and n = p + h is the number of exitons. 

Cascade pre-equilibrium model uses target exitation data, 
exiton configuration for neutron and proton to produce non-equilibrium evaporation.
The angular distribution is isotropic in the frame of rest of the exiton system.

Parametrisations of the level density, is tabulated both with A and Z dependence and with high temperature 
behaviour (the nuclei binding energy using smooth liquid high energy formula).

%N $\rightarrow$ N - 2, N $\rightarrow$ N - 1 N $\rightarrow$ N, N $\rightarrow$ N + 2,  channels are treated.

\subsection{Break-up models}


Fermi break-up is allowed only in some extreme cases, i.e. for light nuclei ($A < 12$ and  $3 (A - Z) < Z < 6$ ) and $E_{exitation} > 3 E_{binding}$ 
Simple explosion model decays the nuleus into neutrons and protons and decreases exotic evaporation processes.


Fission model is phenomenological model using potential minimization. Binding energy paramerization is used and
some features of the fission statistical model are incorporated \cite{fong69}.

\subsection{Evaporation model}

Statistical theory for particle emission of the exited nucleus remaining after the INC was originally developed by Weisskopf \cite{weisskopf37}. 
This model assumes complete energy equilibration before the particle emission, and re-equilibration of excitation energies between successive evaporations. 
As a result the angular distribution of emitted particles is isotropic.

{\sc Geant4} evaporation model for cascade implementation adapts often used computational method developed by Dostrowski \cite{dostrovsky59, dostrovsky60}.
The emission of particles is computed until the exitation energy fall below some spesific cutoff. 
If light nucleus is higly exited Fermi break-up model is executed. Also, fission is performed if channel is open. 
The main chain of evaporation is followed until  $E_{exitation}$ falls below E$_{cutoff}$ = 0.1 MeV. 
The evaporation model ends with  emission chain which is followed until E$_{exitation}$ $<$ E$^{\gamma}_{cutoff}$ = 10$^{-15}$ MeV.

%Evaporation (Weisskopf - Ewing) 
%Weisskopf-Ewing evaporation in competition with fission. 
