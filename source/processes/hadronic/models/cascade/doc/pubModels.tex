\section{Geant4 cascade model}

%     of an intermediate-energy intranuclear cascade and evaporation models. 
[outline, and general description, overview]

Intranuclear cascade (INC) model developed by Bertini \cite{bertini68}, solves the Boltzman equation on the average.
Nuclear model consist of a three-region approximation to the continuously changing density distribution of nuclear matter within nuclei.

INC model had been implemented in several codes such as HETC \cite{alsmiller90}. 
Our model is based on re-engineering of INUCL code~\cite{titarenko99a}.
Models included are Bertini intra-nuclear cascade model with exitons, pre-equilibrium model, simple nucleus explosion model, 
fission model, and evaporation model. 

%Fast pahse of initial INC results to highly exited nucleus -> fission / pre-equilibrium emission -> And is followed by slower compound nucleus phase --> fission/evaportaion
%Protons and neurons are assumed to have a zero-temperature ::: Fermi distribution. Hence, their kinetic energy range from zero to the zero-temperature Fermi energies calculated from the dencity of nucleids in ecah region.
%range of projectiles allowed p, n, pi and nuclei.
%From a few Mev to 10 GeV for n, p, pi and up to about 100 MeV / nucleon for nuclei.

% particles p, n, pi, D, T, He3, He4,\gamma
% (INUCLN with neutrino 31.3.98)
% pion aborption
% pre-equilibrium exiton model 


Cascade is stopped when all the particles, which can escape the nucleus, do it. 
Then conformity with the energy - conservation law is checked and the given event is accepted, if $E_{exitation} > E_{cut} \approx $a few $MeV$.

% Short sketch of the model:

% \begin{itemize}
% \item If an interaction is supposed to happen, the impact point onto thenucleus is calculated
% \item the nucleus is split into three different regions with different potentials, nucleon densities and Fermi energies.

% \item nucleon-projectile interction is chosen according to partial cross-sections
% \item the products of the interaction are tracked trough the regions  until the energy drops below a cut-off energy.
% \end{itemize}

In the following sections we give the overview of submodel basic features.

%Originally the INUCL code was published in 1983 \cite{stepanov}.
%Comprison of results between INCUL and LAHET, CEM95, HETC, CASCADE,
%YIELDX, and ALICE code are presented in Refs. \cite{titarenko99a}.

\subsection{Model limits}

Range of targets allowed is arbitrary.
Reactions from $\approx$ 100~MeV to $\approx$ 5~GeV energy are treated for proton, neutron, pions, photon and nuclear isotopes.

At energies higher than $\approx$ 10 GeV) the INC picture breaks down, also physical foundation comes approximate at energies less than $\approx 200 MeV$, and needs to be supported with pre-quilibrium model.

\subsection{Nuclei model}

\subsubsection{Layers}

Bertini INC is based on layered nuclei model. Our implementation
consists of three concentric spheres, each with uniform density of neutrons and protons.
Nucleon  are supposed to to move in a radius-dependent well which incorporates both muclear and Coulomb effects.

[Nuclear radius]
Nuclear radius parameterization: By the definition R(A) is derived from eq. $Den(R(A)) = 0.01 Den max$.

%Layer radius 0.1 0.2 0.9.

[Nuclear density]
Nuclear density distribution are derived from the $Re(V_{opt}(r))$ distribution. In cascade part, nucleus is divided into a finite number of zones with constant density.
Fermi energy calculated in a local density approximation.
%The nucleons are assumed to to have a Fermi gas momentum distribution.
%The potential well depth is taken to be 7 MeV.

The nucleon potential energy is
$ V_{N} = \frac{p_{f}^2}{2 m_{N}} + BE_{N}(A, Z)$

%vz.push_back(0.5 * pff * pff / mproton + binding_energies[0]);
, where $p_{f} = 1.932 r

%G4double dd0 = 3.0 * z * oneBypiTimes4 / tot_vol;
%G4double rd = dd0 * v[i] / v1[i];
%G4double pff = pf_coeff * pow(rd, one_third);

Nuclear density effects are recalculated after each step.

Impulse distribution

Distribution of potential energy.


[Mass formula]
Nucleons binding energies (BE) are calculated using mass formula.
A parametrization of the nuclear binding energy uses combintaion of Kummel mass formula, or
experimental data. Asymptotic high temperature mass formula is used if it's impossible to use experimental data.
%For pions $V_{opt}$ is taken to be constant (about 7 MeV).
%Smooth liquid high energy formula  has  

% Continuous charge dencity
% \begin{equation}
% \rho(r) = \frac{\rho_1}{e^{\frac{r-c}{a}}+1}
% \end{equation}

% where $a= 0.545~fm$ and $cA^{1/3} = 1.07~fm$. $A$ is mass number and $\rho_1$ mormalization parameter.

% In the code the nucleus is composed of three consentric spheres, 
% corresponding $0.9$, $0.2$, and $0.01$ of the maximal dencity.
% In every reqion the dencity is uniform, 
% and defined as mean value of continuos charge distribution.
% Normalization is set so that when integrating over the reqions, correct number of nucleons is achieved.
 
% \subsubsection{Impulse distribution}
% Impulse distribution in each region follows Fermi distribution with zero temperature.

% \begin{equation}
% f(p) = c p ^2
% \end{equation}

% where

% \begin{equation}
% \int_0^{P_f} f(p) dp = n_p or n_n
% \end{equation}

% where $n_p$ and $n_n$ are the number of protons or neutrons in region.
% $P_f$ is impulse corresponding the Fermi energy


% \begin{equation}
% E_f = \frac{P_f^2}{2m} = \frac{\hbar^2}{2m}(\frac{3\pi^2n}{v})^\frac{2}{3}
% \end{equation}
 
% which depend on the dencity $n/v$ of particles, 
% and which is different for each particle and each region. 
% The total distribution as on composite does not follow Fermi distribution with zero temperature.

% \subsubsection{Distribution of potential energy}
% The binding energy is systematically set to be $7~MeV$. 
% In each region protons and neutrons have different potential energy.  

%\subsubsection{Quantum effects}
\subsubsection{Pauli exclusion principle}
Pauli exclusion principle forbids interactions where the products would bi in occupied states.
Following an assumption od completely degenerate Fermi gas, the levels are filled from the lowes level.
The minimum energy allowed for the product of collision correspond to the lowest unfilled level of system, which is the Fermi energy in the region. 
So in practice, Pauli exclusion prinsiple is taken into account by accepting only  secondary nucleons which have $E_n > E_f$.

% Constrains imposed by the Pauli exclusion principle 
% are taken into account 
% by comparing energies of secondary particles.
% If among socondary particles 
% there is a nucleon with the energy lower than the Fermi energy 
% $E < E_F$, then this interaction is consideres prohibited,
% and the trajectori of the particle is traced further from the forbideden point.


% Energies of particles are compared with the so-called cut-off energy $E_{cutoff}$.
% Particles with the energy $E > E_{cutoff}$ participate in the development of the INC.
% Typical values are listed in Table~\ref{table:cutoff}.


% \begin{table}[!hbt]
% \caption{Typical cut-off values for HETC particles.}

% \hspace{0.5cm}

% \label{table:cutoff} 
% \centering
% \begin{tabular}{lcll}
% \hline
% \em Particel    & \em cut-off           \\
% \em Particel    & \em enrgy [MeV]       \\[2.5ex] 
% \hline
% $p$             & $15$                  \\
% $n$             & $15$                  \\
% $\mu$           & $0.2$                 \\
% $\pi^\pm$       & $2$                   \\
% \hline
% \end{tabular} 
% \end{table}


% \subsubsection{Description of INC}

% A key feature of this model is that, at sufficiently high energies,
% the initial phase of reaction can be treated int terms of collisions
% of the incident particle with individual nucleons inside the nucleus. 
% The struck nucleons can cause further collisions, giving rise to a
% particle ``cascede'' inside the nucleos; hence the term intranuclear
% cascade describing this process.

% \begin{figure}
%   \begin{center}
%     \leavevmode
%     \rotatebox{0}{\mbox{\epsfxsize=8cm \epsfysize=6cm \epsffile{\PIC /mc.eps}}}
%         \caption{Schematic diagram describing Monte Carlo simulation of INC. 400 MeV proton in colliding withnucleus, 
% Here. for simplicity only one region is used. Crosses incicate Pauli-plocking}
%   \label{sibtoo}
%   \end{center}

% \end{figure}


% The implementation of the INC model is briefly described in the following steps:
% \newcounter{list}
% \begin{list}{\upshape \arabic{list}. }
%         {\usecounter{list}
%         \setlength{\labelwidth}{2cm}\setlength{\leftmargin}{2.6cm}
%         \setlength{\labelsep}{0.5cm}\setlength{\rightmargin}{1cm}
%         \setlength{\parsep}{0.5ex plus0.2ex minus0.1ex}
%         \setlength{\itemsep}{0ex plus0.2ex}} %slshape
% \item The spatial point where the incident particle enters the nucleus is determined by selecting it uniformly from the circle presenting the area of the nucleus projected on a plane.
% \item A path length for the distance the particle travels before
%         collision is selected by using the total particle-particle
%         cross-sections and region-dependent nucleon densities.% (three
% %        different regions).
% \item If the particle escapes the nucleus without a collision, it will
%         no longer be tracked in the INC model. Otherwise, the momentum
%         of the struck nucleon, the type of reaction and the energy and
%         direction of the reaction products are determined.
% \item If the collision is not forbidden by the Pauli exclusion principle and
%         if the kinetic energy of the product is above a predefined
%         cutoff energy, the algorithm goes to step 2 to transport all
%         the products further in the nucleus.
% \item When the cascade is completed, the mass $A'$ and charge $Z'$ of
%         the residual nucleus are determined from their conservation
%         laws. Also the residual excitation energy $E^*$ is determined
%         from the energy conservation law.
% \end{list}             
% %\end{minipage}


% After the INC has been accomplished, $A'$, $Z'$ and $E^*$ are
% used as input for the pre-equilibrium and the subsequent evaporation
% algorithms to determine the number, type and energy of nucleons and
% heavier particles ($d$, $t$, :::HET and $\alpha$ particles)
% emitted. The remaining excitation energy is assumed to dissipate by
% photon emission. 
% %The $A$ and $Z$ of the final nucleus are then determined.
% \subsubsection{Scaling model}
% The information (particle types, energies, direction cosines, etc.)
% obtained from the INC can be given as input to scaling routines, 
% that use an extrapolation model to obtain the description of the collision products 
% corresponding to the actual energy or the particle.
% \subsection{Pre-equilibrium model}
% \subsubsection{Separtion between INC and evaporation}

\subsubsection{Cross sections and kinematics}

Cascade model has tabulated cross-sections

Reletivistic kinematics is applied throughout the cascade, with accurate conservation of energy and momentum.

Probabllility for interactions during INC is obtained using free (N, N) cross-sections. 
Angles after collisions are sampled from experimental differential cross sections.
%Cross-sections for production of one or two $\pi$ are included. 

%pion absorption

Path lengths of nucleons in the nucleus are sampled according to the local density and to free N-N cross sections.
%Total inelastic cross section has to be taken from outside to normalize all data. 
% \item Total reaction cross-sections [mbarn] were calculated by J.R. Letaw's formulae 
%$45 A^0.7 (1+0.016 sin(5.3-2.63 log10(A)))^(1-0.62 exp(-E / 200) sin(10.9 E^(-0.28)))
%S. Pearlstein, The Astrophysical Journal, 346: 1049-1060, 1989 November 15

%Nucleon nucleon cross-sections:
%Parametrizations based on the experimental data (ED) are used. 
%They are energy and isospin dependent. 
%The parameterizations described in ([1] Barashenkov V.S., Toneev V.D. High Energy interactions of particles and nuclei with nuclei. Moscow, 1972 
%(in Russian, but there is an English translation)) are used.

% interaction crosssections
% (all data for (N, N) and (pi, N) interactions (dn/dsigma, d3sigma/d3p, 
%             partial multiplicity for npi<=5  error 10-20\%)



\subsubsection{Exiton model}

% %What criteria for p-h excitation? is the next phase precompound or compound'? Only pions. The next phase is precompound. 
%Initial conditions are defined during the cascad phase: 
%p -number of "particles", i.e. nucleons, which can not escape the nucleus and have too small interaction probability; 
%h - number of "holes" = number of nuclear nucleons involved in the cascade; energy - momentum of the exiton system derived from the conservation law.

%exiton master equation


% the Monte Carlo calculation of nuclear
% reactions in the framework of the Cascade-Exciton Model (CEM) of
% nuclear reactions. The CEM assumes that reactions occur in three
% stages. The first stage is the intranuclear cascade. The excited
% residual nucleus formed after the  emission of cascade particles
% determines the particle-hole configuration that is a starting point
% for the second preequilibrium stage of the reaction. The subsequent
% relaxation of the nuclear excitation is treated in terms of the
% exciton model of preequilibrium decay which includes the description
% of the equilibrium evaporative stage of the reaction.


%  The CEM95 code is intended for calculation of reaction, elastic,
%  fission and total cross sections;  excitation functions; nuclide
%  yields, energy and angular spectra; double differential cross
%  sections; mean multiplicities, i.e., number of ejectiles per incident
%  bombarding particle; ejectile yields; mean energies and production
%  cross sections for neutrons, protons, deuterons, tritons, He3, He4,
%  pions-, pions0, and pions+ emitted in nucleon- and pion-induced
%  reactions using the Cascade-Exciton Model (CEM) of Nuclear
%  Reactions. A detailed description of the CEM may be found in
%  Ref.1. Part of primary version of the code concerning the
%  preequilibrium and equilibrium stages of reactions is published in
%  Ref.2. The Dubna version of the  intranuclear cascade model used in
%  the CEM95 is described in detail  in the monograph 3. A detailed
%  description of the subroutines used at the cascade stage of reaction
%  may be found in Ref.4. All the models incorporated in the CEM95 for
%  description of the level density parameter are given in Ref.5. All
%  the models incorporated in the CEM95 to take into account competition
%  between particle emission and fission at the compound stage of the
%  reactions are described in  Ref.6. Exemplary results obtained with
%  the code CEM95 may be found in Refs.7,8.

% 9.
% STATUS
% IAEA1247/01: 21-APR-1995 tested at NEADB
% 1015.

% NAME�AND�ESTABLISHMENT�OF�AUTHORS�-.

% ��Dr.�Mashnik�Stepan
% ��Laboratory�of�Theoretical�Physics
% ��Joint�Institute�for�Nuclear�Research
% ��141980�DUBNA
% ��Moscow�Region
% ��RUSSIAN�FEDERATION

% A.  Cross Section and Resonance Integral Calculations                              
% Computation of reaction  cross sections  from nuclear theory such  as        the  optical  or  Hauser Feshbach models,  resonance cross  sections  by        Breit Wigner or  multilevel theory, determination of differential  cross sections, cross section evaluation, and compilation programs.
% Keywords: evaporation model, high-energy reactions, nuclear cascades,
% nuclear models. \cite{gudima83}, \cite{mashnik94} 

% \begin{itemize}

% \item 
% %\item CEM95 http://www.nea.fr/abs/html/iaea1247.html, http://www.nea.fr/abs/html/iaea1247.html}

% \end{itemize}

\subsection{Pre-equilibrium model}

Uses target exitation data, exiton configuration for neutron and proton to produce non-equilibrium evaporation.
Implementaion inclues holes and quasiparticles.

% \subsubsection{Precompound phase}
% %, describe the PE model used, parameters, i.e., partial state densities, transition rates? 

% Parametrixations of the level density, tabulated both with A and Z dependence and with high temperature 
%behaviour (the nuclei binding energy using smooth liquid high energy formula)

%Main parameters are taken from (Ribansky I. et al, Nucl.Phys.,1973, A205, p.545 (level densities); 
%Kolbach.C., Z.Phys.,1978, A287, p.319 (matrix elements)). 
%N -> N, N -> N + 2, N -> N -2, N -> N - 1 channels are treated.)
%The angular distribution is isotropic in the frame of rest of the exiton system.

% The outline of the pre-equilibrium model is as follows~\cite{ferrari96}:
% \begin{list}{\upshape \arabic{list}. }
%         {\usecounter{list}
%         \setlength{\labelwidth}{2cm}\setlength{\leftmargin}{2.6cm}
%         \setlength{\labelsep}{0.5cm}\setlength{\rightmargin}{1cm}
%         \setlength{\parsep}{0.5ex plus0.2ex minus0.1ex}
%         \setlength{\itemsep}{0ex plus0.2ex}} %slshape
%     \item For a given excitation energy $E$, atomic number $A$ and
%       number of excitons $n$ the equilibrium number of excitons
%       $n_{eq}=\sqrt{0.5 + 2 g E}$ is calculated ($g$ is the single
%       particle level density). Now if the current exciton number
%       $n\geq n_{eq}$ or the excitation energy is below a predefined
%       level (i.e. 50~MeV), then the pre-equilibrium algorithm is
%       finished and further emission of fragments is done with an
%       equilibrium model (here the evaporation model). If $n<n_{eq}$,
%       the next step is performed.
% \item   The transition probabilities and emission probabilities for $n$, $p$, 
%         $d$, $t$, :::HET and :::HEF are calculated for
%         the current nucleus. The probabilities are normalized, and one
%         subprocess (emission or transition) is selected.
% \item   In case of a transition the number of excitons is updated, if
%         necessary, and the algorithm returns to step~1. Otherwise 
%         the next step is performed.
% \item   The kinetic energy of the fragment is selected.
% \item   Momentum direction angles of the fragment are sampled from an
%         isotropic distribution in the exciton system rest frame. The
%         momentum is then boosted to the nucleus rest frame.
% \item   Characteristics of the residual nucleus ($A$, $Z$, $E$ and 
%         momentum $P$) are updated. The algorithm returns to step~1.
% \end{list}             
% %\end{minipage}

% In HETC pre-equilibrium code have been used to replace the INC in in energy range $E < 200~MeV$.

\subsection{Fermi break-up model}

Fermi break-up is allowed only in some extreme cases, i.e. for light nuclei ($A < 12$ and  $3 (a - z) < Z < 6$ ) and $E_{exitation} > 3 E_{binding}$ 
%If the light nuclei exitation energy is higher than three times the binding energy, 
%the nuclei expolision is performed. 

Simple explosion model decays the nuleus into neutrons and protons and decreases exotic evaporation processes.

\subsection{Fission model}

Fission model is phenomenological, incorporating some features of the fission statistical model.

\subsection{Evaporation model}

Physics of particle emission of the exited nucleus (de-exitation) remaining after the INC is based on statistical theory
originally developed by Weisskopf \cite{weisskopf37}. This model assumes complete energy equilibration before the particle emission, and re-equilibration of excitation energies between successive evaporations. As a result the angular distribution of emitted particles is isotropic.

{\sc Geant4} evaporation model for cascade implementation adapts often used comutational method developed by Dostrowski \cite{dostrovsky59, dostrovsky60}.
The emission of particles is computed until the exitation energy

%Evaporation (Weisskopf - Ewing) 
%Weisskopf-Ewing evaporation in competition with fission. 
%Emissions of n,p,d,t,He3,He4,gamma is allowed. 
%Level densities derived from exp.data are used. 
%Angular momentum and spin dependence are not included. 

%Parameters used: level densities, inverse cross-sections or transmission coefficient, 
%choice of optical model parameters if relevant (or reference to source), 
%range of excitations allowed, inclusive or exclusive results? 

%Other parameters are the same as in ([1], see 5a.) 

% \subsection{Evaporation model}

% After the pre-equilibrium model, the nucleus is assumed to be in an
% excited state, but in an internal thermal equilibrium.  The excitation
% leads to emission of nucleons, light nuclei and $\gamma$ particles,
% which are handled with the evaporation and de-excitation algorithms
% \cite{iljinov92}.

% %described in {\bf Chapter}. 

% The evaporation algorithm %described in Fig.~\ref{fig:hetcVuokaavio} 
% %is quite straight-forward. It 
% consists of subsequent evaporation cycles, which are repeated until no
% further evaporation is energetically possible.
% %as long as the nucleus remains excited. 
% On
% each evaporation cycle, one particle ($n$, $p$, $d$, $t$, :::HET or
% :::HEF) is sampled and emitted. The emission probabilities are obtained
% according to the statistical evaporation theory of
% Weisskopf~\cite{weisskopf37},where the probability to emit a
% particle $i$ with kinetic energy between $\epsilon$ and $d\epsilon$
% from a compound nucleus excited to an energy $E^*$ (measured from the
% ground state) is
% \begin{equation}\label{eq:weisskopf}
%         P_i(\epsilon)d\epsilon = \frac{(2S_i+1)m_i}{\pi^2 \hbar^3} \sigma_{inv,i}(\epsilon)
%         \frac{\rho_f(E^*- Q_i - \epsilon)}{\rho_i(E^*)}\epsilon d\epsilon,
% \end{equation}
% where $\rho$'s are the nuclear level densities for final and the
% initial nuclei, $\sigma_{inv,i}$ is the cross-section for the inverse
% reaction (total cross-section for capture of particle $i$ of energy
% $\epsilon$ by the residual nucleus), $S_i$ and $m_i$ are spin and mass
% of the emitted particle and $Q$ is its separation energy. This formula
% also describes the kinetic energy distribution of the emitted particle.  The
% state of the residual nucleus is updated when another cycle begins.
% When no further particle emission is possible, the (still excited)
% residual nucleus is given to the de-excitation algorithm. It is also
% possible that there is no residual nucleus, as in the splitting of
% :::BEE.

% In the de-excitation algorithm, the remaining excitation energy of the
% residual nucleus is used to emit photons, which are uniformly
% distributed in the CMS frame. Their kinetic energy is sampled from a
% uniform distribution between zero and the residual excitation energy.
% Photons from this energy distribution are emitted as long as the
% remaining excitation energy is positive; otherwise the kinetic energy
% of the emitted photon is greater than the remaining excitation energy,
% and in this case it is set to be equal to the remaining excitation
% energy, which then goes to zero. This also ends the de-excitation
% algorithm.








