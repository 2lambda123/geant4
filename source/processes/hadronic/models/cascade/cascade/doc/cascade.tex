\documentclass{hep99}
%\usepackage{epsfig}
\usepackage[dvips]{graphicx}
\begin{document}
\title{Implementation of High Energy Transport Code (HETC) cascade
model into GEANT4}
\author{J. Chuma, A.~Heikkinen}
\address{Helsinki Institute of Physics, University of Helsinki,
Finland\\[3pt]
E-mail: {\tt Aatos.Heikkinen@cern.ch}}

\abstract{
We present description of HETC model and discuss the implementation of
this intranuclear transport code into GEANT4.
}
\maketitle
\section{Preface}
\cite{geant4}

The structure of this paper is as follows:
section~\ref{sec:intro} gives an introduction to HETC model
section~\ref{sec:theory} discusses the HETC model
section~\ref{sec:design} discusses the analysis and design in GEANT4
section~\ref{sec:implementation} describes implementation issues and
section~\ref{sec:comparison} gives comparisons of simulation and
measured data
section~\ref{sec:conclusion} provides  overview of the project status.

%-----------------------------------------------------------------------------
\section{Introduction to HETC\label{sec:intro}}
%-----------------------------------------------------------------------------
\section{HETC model\label{sec:theory}}
%------------------------------------------------------------------------
\section{OO Analysis and design of HETC++\label{sec:design}}

\ref{fig:uml} 

%-----------------------------------------------------------------------
\section{Implementation details\label{sec:implementation}}
The Rational Rose CASE tool was used to generate source code directly
from UML diagrams. 

We use the tools laif by GEANT4 and LHC++ \cite{lhc++}.
%------------------------------------------------------------------------
\section{Conclusion\label{sec:conclusion}}

\begin{thebibliography}{9}
\bibitem{geant4} GEANT4: LCB Status Report/RD44 1998 {\it CERN/LHCC-98-44} 
\bibitem{lhc++} LCB Status Report/LHC++ 1998 {CERN/LHC 98-11}
\bibitem{cms} Muller T 1998 {\it Nucl. Instrum. Meth.} A {\bf 408} 119


\end{thebibliography}

%-----------------------------------------------------------------------

\begin{figure}
\begin{center}
%\includegraphics[width=2.5in]{uml.ps}
\end{center}
\caption{xxx}
\label{fig:uml}
\end{figure}

%If the figure is too large to fit in a column a * is added after
%\verb"figure" to use the \verb"figure*" environment. Figures should be
%centred in the column or page.


\begin{figure}
\begin{center}
%\includegraphics[width=3in]{finder.ps}
\end{center}
\caption{xxx.}
\label{fig:finder}
\end{figure}

\end{document}




