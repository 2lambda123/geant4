\documentclass[12pt]{article}

\oddsidemargin  0.0cm
\evensidemargin 0.0cm

\topmargin 0cm
\headheight 0pt
\headsep 0pt
\topskip 0pt
%\footheight 1cm
%\footskip 1cm

%\textheight 24.0cm
%\textwidth 16.0cm
%\parindent 7mm
%\parskip 0pt
%\def\DD {{\!}_{DD}}

\tolerance = 400
\begin{document}

{\Large{\bf The generator of events of a hadron-nucleus
elastic scattering at high energy.}}

\vspace{2ex}
{\bf The  calculations method.}
%\par
\vspace{2ex}

For the calculations  the Glauber model was used. This model describes the 
hadrons-nucleus elastic and difraction scattering rather well
at medium and high energy. 
The corrections on 
the inelastic screening was taken into account too. 

The important difference from usual calculations is that the two-gaussian 
form of a nuclear density was used


  $$
   \rho(r) = C (e^{-(r/R_{1})^2}-pe^{-(r/R_2)^2}),   
   \eqno (1)
  $$
where $R_1$, $R_2$ and $p$ are the fitting
parameters but C is the normalization constant.


This representation of density allows to get the expressions 
for an amplitude in the analitic form.  

The form (1) is not physical especially for a heavy nucleus. 
The reason why it works rather well (see below the figures of
descriptions of  experimental datas) is that the nucleus absorbs 
of the hadrons very strongly especially at small impact parameters 
where the absorption is full. So the representation (1) describes 
only the peripherial edge of a nucleus.

 We use an usual expression for the  Glauber model amplitude 
of a multiple scattering 
 

  $$
   F(q)=\frac{ik_{CM}}{2\pi} \int d^2{b}e^{\vec{\mathstrut q}
   \cdot \vec{\mathstrut b}} M(\vec b).
   \eqno (2)
  $$
Here $M(\vec b)$ is the hadron-nucleus amplitude in the impact 
parameter representation
  $$
 M(\vec b) = 1-[1-e^{-A\int d^{3}r\Gamma(\vec{\mathstrut b}-
      \vec{\mathstrut s})\rho(\vec {r})}]^A,
 \eqno (3)
  $$
$k_{CM}$ is an incident particle momentum,
$\vec q$ is a transfer momentum in the center of mass system
and $\Gamma(\vec {b})$ is the hadron-nucleon amplitude of 
the elastic scattering in the impact-parameter 
representation

  $$
  \Gamma(\vec b)=\frac{\displaystyle 1}
                      {\displaystyle 2\pi ik_{CM}}
  \int d\vec {q} e^{-\vec {\mathstrut q} \cdot \vec{\mathstrut b}} 
  f(\vec {q}).
  \eqno (4)
  $$

We use the exponential parametrisation of the hadron-nucleom
amplitude

  $$
  f(\vec {q})=\frac{\displaystyle ik_{CM}\sigma^{hn}}{2\pi}
   e^{-0.5q^2B}.
  \eqno (5)
  $$
Here  $\sigma^{hn}=\sigma_{tot}^{hn}(1-i\alpha)$, 
$\sigma_{tot}^{hn}$ is the total
cross-section hadron-nucleon scattering, B is 
the difraction cone slope  
and $\alpha$ is the relation of a real to
imagenary part of amplitude at $q=0$. 

 Substituting (1) and (5) into (2), (3) and (4)
we can easy get the following formula
 
 $$
\begin{array}{cl}
   F(q) = \frac{\displaystyle ik_{CM}\pi}{\displaystyle 2} 
   \sum\limits_{k=1}^A (-1)^k {\displaystyle A\choose 
   \displaystyle k} 
   [\frac{\displaystyle\sigma^{hn}}
   {\displaystyle 2\pi(R_1^3-pR_2^3)}]^k 
   \sum\limits_{m=0}^{k} (-1)^m {\displaystyle k \choose 
   \displaystyle m}
   \left[ \frac{\displaystyle R_1^3}{\displaystyle R_1^2+2B}
   \right]^{k-m} \\
  \qquad {} \left[ 
  \frac{\displaystyle pR_2^3}{\displaystyle R_2^2+2B}
  \right]^{m} \left( 
  \frac{\displaystyle m}{\displaystyle R_2^2+2B}+
  \frac{\displaystyle k-m}{\displaystyle R_1^2+2B} \right)^{-1}\\
%  \eqno (6)

  \qquad {} \exp\left[ -\frac{\displaystyle -q^2}{\displaystyle 4} 
  \left(
  \frac{\displaystyle m}{\displaystyle R_2^2+2B}+
  \frac{\displaystyle k-m}{\displaystyle R_1^2+2B} \right)^{-1}
  \right].
\end{array}
  \eqno (6)
 $$

The analogous procedure can be used to get the inelastic 
screening corrections to the hadron-nucleus amplitude
$\Delta M(\vec b)$.
In this case the intermediate inelastic difractive
 state is
created, rescatters on the nucleus nucleons and then
returns into the initial hadron. We do not present
the expressions for the corresponding amplitude
because it is quite long.

The full amplitude is the sum $M(\vec b)+
\Delta M(\vec b)$.

The differencial cross-sections are connected with 
the amplitude in the following way


$$
  \frac{d\sigma}{d\Omega_{CM}}=\left| F(q) \right|^2, \qquad 
   \frac{d\sigma}{|dt|}=\frac{d\sigma}{dq_{CM}^2}=
  \frac{\pi}{k_{CM}^2} \left|F(q) \right|^2.
   \eqno (7)
$$   
                      
The  fitting of the parameters of nuclei densities was made 
in a wide area of  atomic
numbers $(A=12 \div 208)$ using experimental data on a 
proton-nuclei elastic scattering at a proton 
kinetic energy $T_p=1GeV$. The proton-nucleon parameters
at this energy were fixed. The fitting was perfomed both for
the individual nucleus and for all set of nuclei at once. 
In the last event the following dependensies for nuclei
parameters were obtained

\vspace{2ex}
$R_1=4.18A^{0.302}$, $R_2=3.81(A-10)^{0.268}$, $p=0.95$.
\hspace{3cm} (8)
\vspace{2ex}

It is necessary to note that for  every nucleus 
the optimal set of density parameters exists and it
difers slightly from one derived from (8).

The comparision of the phenomenological cross-sections (6) with the 
experiment is represented on the fugures from 1 to 8
where the data of the proton-nucleus scattering 
are imaged.  The dependence (8) was used for nuclei parameters.
The proton kinetic energy  is $T_p=1 GeV$ in the laboratory. 
The experimental datas was obtaned 
in Gatchina (Russia) and in Sacle (France). 
The horizontal axes is  $q^2_{CM}$ in $GeV^2$ 
but the vertical one is
$\frac{\displaystyle d\sigma}{\displaystyle d\Omega_{CM}}$ 
in $\frac{\displaystyle mb}{\displaystyle Ster}$.
The lower curve in the figures is the diferencial cross-section 
of a coherent elastic scattering bur upper one is the
cross-section for noncoherent scattering.

The creating of generator of events claims the knowing of the
distribution function ${\cal F}$ of a corresponding process. 
The diferential
cross-section is proportional to the density distribution.
Therefore to get the distribution function it is enough to
take an integral of differential cross-section and to
normalize it

 $$
    {\cal F}(q^2)=
 \frac
 {\displaystyle
   \int\limits_{0}^{q^2}d(q^2)
   \frac {\displaystyle d\sigma}{\displaystyle d(q^2)}
  }
 {\displaystyle
   \int\limits_{0}^{q_{max}^2}d(q^2)
   \frac {\displaystyle d\sigma}{\displaystyle d(q^2)}
  }
  \eqno (8)
  $$

  The expressions (6) and (7) allows the analitic integration
in (8) but the result is very long to be writen here.

For light nuclei the analitic expression is more convinient
in calculations
in comparision with the numeric integration in (8) but for
heavy nuclei last is more prefered due of large number
of members in the analitic expression for them.

\vspace{2ex}

{\bf The description of the classes for an elastic hadron-nucleus 
scattering at high energy.}\\
\par
\vspace{1ex}

The set of classes for calculation of the elastic coherent processes includes 
the following classes:

\begin{enumerate}

\item {\bf G4HadronValues.} This class allows to get 
the hadron-nucleon 
parameters (the total cross-sections, the slope parameter and 
the relation a real to an imaginary 
part of amplitude). It has the constructor without parameters and the member-
function {\tt void GetHadronValues(const G4DynamicParticles *)}, 
which calculates  
the protected  member-values (all {\tt G4double}) 
{\tt HadrTot},  {\tt HadrSlope},  {\tt HadrReIm}. 
These values are the total cross-section in mb, the slope parameter in   
$GeV^{-2}$ and 
the relation a real to an imaginary part of an elastic amplitude, 
correspondingly for 
hadron-nucleon scattering at an energy included in 
{\tt G4DynamicParticle}.

\item   {\bf G4IntegralHadrNucleus.} This class calculates an integral
hadron-nucleus cross-sections 
and is derived class of the {\tt G4HadronValues}. 
It  has  the constructor
without arguments and five public member-functions 
which return {\tt G4double} values:

\begin{enumerate}
%\begin{itemize}
 \item {\tt GetTotalCrossSection(G4DynamicParticle *, G4Nucleus *).}
 
This function returns the corresponding total cross-section in mb.

 \item {\tt GetProductionCrossSection(G4DynamicParticle *, G4Nucle\-us
*).} 

This function returns the corresponding production cross-section in mb.

 \item {\tt GetElasticCrossSection(G4DynamicParticle *, G4Nucleus *).} 

This 
function returns the corresponding elastic cross-section in mb.

 \item {\tt GetInelasticCrossSection(G4DynamicParticle *, G4Nucleus *).} 
This 
function returns the corresponding inelastic cross-section in mb.

 \item {\tt GetQuasyElasticCrossSection(G4DynamicParticle *, 
G4Nu\-cle\-us *).} 
This function returns the corresponding quasyelastic \
cross-section in mb.

\end{enumerate}
%\end{itemize}

\item  {\bf G4DiffElasticHadrNucleus.} This class  calculates the
differential cross-section of an elastic 
hadron-nucleus scattering. This class is derived of the 
{\tt G4HadronValues} and has the
constructor without of arguments. It has one member-function 
{\tt HadronNucleusDifferCrSec(G4Dyna\-micPar\-ti\-cle *, G4Nucleus *,
G4double)} 
which returns the value ({\tt G4double}) of the differential 
cross-section 
$\frac{\displaystyle d\sigma}{\displaystyle d\Omega_{CM}} $
in $ \frac{\displaystyle mb}{\displaystyle Ster}$
for the momentum transfer squared $q^2$ (in $MeV^2$) 
which is equal the third argument. 

\item {\bf G4ElasticHadrNucleusHE} is the generator of 
events for an elastic hadron-nucleus 
scattering. This class has five overloaded constructors 
with different sets of arguments.

\begin{enumerate}
%\begin{itemize}
%\begin{list}%{\alph}{}
%{\usecounter{tmp}}

\item {\tt G4ElasticHadrNucleusHE(G4DynamicParticle *, 
G4Nucleus *)}. 
This constructor creates the generator of the events 
at the fixed kinematic variables of an incident particles. 
 The member-function {\tt RandomElastic0(G4DynamicParticle *, 
G4Nucleus *)} returns the transfer momentum  $q_{CM}^2$
(in $MeV^2$)in the center of mass system. 
The distribution density of generated events 
is proportional to the differential 
cross-section at the energy included in {\tt G4DynamicParticle}.

\item {\tt G4ElasticHadrNucleusHE(G4DynamicParticle *, 
G4Nucleus *,
G4double, G4double, G4int, G4String)}. This constructor creates 
the generator of the events which works in user defined 
energy range. The third and fourth arguments 
describe the energy interval where the events will be generated. 
When the class is created the two-dimensional (on $E$ and $q^2$)
array is formed for the distribution function and is 
written in the file under file-name noticed by the sixth argument.  
The number of points on energy (including the beginning end the end) 
is equal to the fifth argument.
The member-function {\tt RandomElastic1(G4DynamicParticle *,
G4Nucleus *, G4double, G4double, G4int, G4String)} returns
the transfer momentum $q^2$ ({\tt G4double}) in $MeV^2$. 

Besides, the created class allows to realize the tracking 
with the calling the member-function 
{\tt ApplayYourself(const G4Track \&, G4Nucleus \&)} which returns
the reference on class {\tt G4VParticle\-Chan\-ge}.
In this case the recoil nucleus  is included in 
the {\tt G4Track} as the secondary particles.

\item  {\tt G4ElasticHadrNucleusHE(G4DynamicParticle *, 
G4Nucleus *, 
G4String)}. This constructor is analogues to the previous constructor. 
The difference is that in this case the two-dimensional 
array is read from the file with name included in 
the third argument. Obviously that this file must be created with 
the previous constructor.

\item {\tt G4ElasticHadrNucleusHE(G4DynamicParticle *, 
G4Nucleus *,
G4double, G4double, G4int)}. This constructor creates 
the generator of the events which works in wide range of hadron 
energy: $E=2 GeV \div 100 TeV$.  
When the class is created the two-dimensional (on $E$ and $q^2$)
array is formed for the distribution function and is 
written in the file whose name is formed from the name of hadron,
the underscore, the nucleus atomic number 
and with the extension ".dat"(for example 
"PionPlus\underline{ }64.dat" for $\pi^+Cu$ scattering). The file is
written into subdirectory "Elastic".

The member-function {\tt RandomElastic1(G4DynamicParticle *,
G4Nucleus *, G4double, G4double, G4int, G4String)} returns
the transfer momentum $q^2$ ({\tt G4double}) in $MeV^2$. 

Besides, the  created class allows to realize the tracking 
with the calling the member-function 
{\tt ApplayYourself(const G4Track \&, G4Nucleus \&)}.  
The recoil nucleus  is included in 
the {\tt G4Track} as the secondary particles.

\item  {\tt G4ElasticHadrNucleusHE(G4DynamicParticle *, 
G4Nucleus *, 
G4int)}. This constructor is analogues to  previous one. 
The difference is that in this case the two-dimensional 
array is read from the file whose name is formed from names of
hadron and nucleus as described above. Obviously that this file 
must be created with the previous constructor.

\end{enumerate}
%\end{itemize}
%\end{list}

\end{enumerate}


  The testing programs of last class have names {\tt MainForRandom1},
{\tt MainFor\-Ran\-dom2}, {\tt MainForRandom3}, {\tt MainForRandom4} 
and {\tt MainForRandom5} 
for the constructors a, b, c, d and e correspondingly. 

In the Appedix we represent the  program text for 
{\tt MainForRandom4} in which the class
{\tt G4ElasticHadronNucleusHE} is created, 
the result is written into
the subdirectory "Elastic" (note that this subdirectory must
by prepared before working)  and 500000 events are 
randomized with the member-function {\tt RandomElastic1}. 
The result of randomization (the set of $q^2$) 
is written in file "q2from4.dat".
The figure 9 represents the result of the generation of
500000 events of elastic scattering nucleon 
on the nucleus $^{90}Zr$ at energy $E_p=100 GeV$.
\vspace{2ex}

{\bf  Appendix.}

\vspace{2ex}
{\tt
// //  The main program for "G4ElasticHadrNucleusHE" class

// //  using the fourth type of constructor


\#include "G4Proton.hh"

\#include "G4DynamicParticle.hh"

\#include "G4ParticleChange.hh"

\#include "G4Track.hh"

\#include "G4ThreeVector.hh"

\#include "G4Nucleus.hh"

\#include "G4IonConstructor.hh"

\#include "G4ElasticHadrNucleusHE.hh"

\#include "g4std/fstream"

 int main()


 \{

\hspace{1cm}  G4IonConstructor Ion;

\hspace{1cm}  Ion.ConstructParticle();

\hspace{1cm}  G4double        Q2, Momentum;

\hspace{1cm}  G4double        px = 0;

\hspace{1cm}  G4double        py = 0;

\hspace{1cm}  G4double        pz = 0;

\hspace{1cm}  G4ThreeVector   inVector(px, py, pz);

\hspace{1cm}  G4ThreeVector   outVector, aPosition(0., 0., 0.);

\hspace{1cm}  G4PionPlus        *  aPionP    = G4PionPlus::

\hspace{5cm}              PionPlusDefinition();

\hspace{1cm}  G4PionMinus       *  aPionM    = G4PionMinus::

\hspace{5cm}            PionMinusDefinition();

\hspace{1cm}   G4KaonPlus        *  aKaonP    = G4KaonPlus::

\hspace{5cm}              KaonPlusDefinition();

\hspace{1cm}    G4KaonMinus       *  aKaonM    = G4KaonMinus::

\hspace{5cm}          KaonMinusDefinition();

\hspace{1cm}   G4Proton          *  aProton   = G4Proton::Proton();

\hspace{1cm} G4AntiProton      *  aProtonA  = G4AntiProton::AntiProton();

\hspace{1cm} G4DynamicParticle *  aParticle = new G4DynamicParticle;

%\hspace{1cm} G4VParticleChange *  aChange   = new G4VParticleChange;

\hspace{1cm} G4Nucleus            aNucleus, * pNucl;

\hspace{1cm}  const G4double N = 90, Z = 45;

\vspace{2ex}
\hspace{2cm}  aParticle->SetDefinition(aProton);

\hspace{2cm}  aNucleus.SetParameters( N, Z);

\hspace{2cm}        pNucl   = \&aNucleus;

%\hspace{2cm}        G4double aTime = 0.1;

\hspace{2cm}       Momentum = 5000;

\hspace{2cm}        inVector.setZ(Momentum);

\hspace{2cm}     aParticle->SetMomentum(inVector);

\hspace{2cm}  G4ElasticHadrNucleusHE    aElasticRandom(aParticle, pNucl,

\hspace{4cm}     2000., 10000., 5);

\hspace{2cm}  G4cout<<G4endl<< " The array is created !!! "<<G4endl;

\vspace{2ex}

\hspace{2cm}  for(G4int i1=1; i1<=2; i1+=5)

\hspace{2cm}   \{

\hspace{3cm}   Momentum = i1*100000;    // in MeV !!!

\hspace{3cm}   inVector.setZ(Momentum);

\hspace{3cm}   aParticle->SetMomentum(inVector);

\hspace{3cm}   G4std::ofstream TestFile("q2from4.dat", 

\hspace{5cm}            G4std::ios::out);

\hspace{3cm}   TestFile.precision(9);

\hspace{3cm}    TestFile.setf(G4std::ios::scientific);

\hspace{3cm}       for(G4int i2=1; i2<500001; i2++)

\hspace{3cm}     \{

\hspace{4cm}   Q2   = aElasticRandom.RandomElastic1(

\hspace{5cm} aParticle,  pNucl);    //   Q\^2 in MeV\^2 !!!

\hspace{4cm}       TestFile<<"  "<<Q2<<G4endl;

\hspace{3cm}      \}  //  i2

\hspace{2cm}        \}      //  i1

\hspace{1cm}  TestFile.close();

\hspace{1cm}    return 0;

\hspace{0.1cm}  \}
	
%\}

\end{document}

